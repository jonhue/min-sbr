\documentclass{beamer}
%
% Choose how your presentation looks.
%
% For more themes, color themes and font themes, see:
% http://deic.uab.es/~iblanes/beamer_gallery/index_by_theme.html
%
\mode<presentation>
{
  \usetheme{default}      % or try Darmstadt, Madrid, Warsaw, ...
  \usecolortheme{default} % or try albatross, beaver, crane, ...
  \usefonttheme{default}  % or try serif, structurebold, ...
  \setbeamertemplate{navigation symbols}{}
  \setbeamertemplate{caption}[numbered]
  \setbeamertemplate{footline}[frame number]
  \setbeamertemplate{itemize items}[circle]
  \setbeamertemplate{theorems}[numbered]
  \setbeamercolor*{structure}{bg=white,fg=blue}
  \setbeamerfont{block title}{size=\normalsize}
}

\newtheorem{proposition}[theorem]{Proposition}
\theoremstyle{definition}
\newtheorem{algorithm}[theorem]{Algorithm}
\newtheorem{idea}[theorem]{Idea}

\usepackage[english]{babel}
\usepackage[utf8]{inputenc}
\usepackage[T1]{fontenc}
\usepackage{aligned-overset}
\usepackage{alltt}
\usepackage{amsmath}
\usepackage{csquotes}
\usepackage{multicol}
\usepackage{stmaryrd}
\usepackage{tabularx}

\renewcommand\tabularxcolumn[1]{m{#1}}
\newcolumntype{R}{>{\raggedleft\arraybackslash}X}

\usepackage{tikz}
\tikzstyle{node}=[fill=white, draw=black, shape=circle, minimum size=0.75cm]
\tikzstyle{blue node}=[fill=white, draw=blue, shape=circle]
\tikzstyle{red node}=[fill=white, draw=red, shape=circle]
\tikzstyle{blue edge}=[-, draw=blue]
\tikzstyle{red edge}=[-, draw=red]
\pgfdeclarelayer{nodelayer}
\pgfdeclarelayer{edgelayer}
\pgfsetlayers{nodelayer,edgelayer}

\def\code#1{\texttt{\frenchspacing#1}}
\def\padding{\vspace{0.5cm}}
\def\spadding{\vspace{0.25cm}}
\def\b{\textcolor{blue}}
\def\r{\textcolor{red}}

% fix for \pause in align
\makeatletter
\let\save@measuring@true\measuring@true
\def\measuring@true{%
  \save@measuring@true
  \def\beamer@sortzero##1{\beamer@ifnextcharospec{\beamer@sortzeroread{##1}}{}}%
  \def\beamer@sortzeroread##1<##2>{}%
  \def\beamer@finalnospec{}%
}
\makeatother

\usepackage[sorting=ynt,style=alphabetic]{biblatex}
\addbibresource{sources.bib}

\title[Sorting by Reversals]{Sorting by Reversals}
\author{Jonas Hübotter}

\begin{document}

\begin{frame}
  \titlepage
\end{frame}

% Uncomment these lines for an automatically generated outline.
\begin{frame}{Outline}
 \tableofcontents
\end{frame}

\section{Motivation}

\subsection{Symmetric group}

\begin{frame}

\begin{definition}
We define the \b{symmetric group} $\langle S_n, \circ \rangle$ as the group whose elements are all bijections over $[1,n]$ \pause with
\begin{align*}
    S_n = \{(0\ \pi_1\ \dots\ \pi_n\ n+1) \mid \{\pi_1, \dots, \pi_n\} = [1,n]\}
\end{align*}
where $\pi_i = \pi(i)$, $\pi_0 = 0$, and $\pi_{n+1} = n+1$.\pause\spadding

$\pi \in S_n$ is a \b{permutation}.\pause\spadding

$id = (0\ 1\ \dots\ n\ n+1) \in S_n$ is the \textit{identity permutation}.
\end{definition}

\end{frame}

\begin{frame}

\begin{definition}
A \b{reversal} $\rho(i,j) \in S_n$ is defined as
\[
    \rho(i,j) = (0\ 1\ \cdots\ i-1\ \r j\ \r{j-1}\ \r \cdots\ \r{i+1}\ \r i\ j+1\ \cdots\ n\ n+1)
\]
for some $j \geq i$.
\end{definition}\pause

\begin{exampleblock}{Example}
Let $\pi = (0\ 1\ 3\ 4\ 2\ 5) \in S_4$. \par
Then
\begin{align*}
    \pi \circ \rho(2,4) = (0\ 1\ \r{2\ 4\ 3}\ 5).
\end{align*}
\end{exampleblock}

\end{frame}

\subsection{Reversal distance problem}

\begin{frame}

\begin{definition}[reversal distance problem]
Given two permutations $\sigma, \tau \in S_n$ find a sequence of reversals $\rho_1, \dots, \rho_d \in S_n$ such that
\begin{align*}
    \sigma \circ \rho_1 \circ \cdots \circ \rho_d = \tau
\end{align*}
and $d$ is minimal. \pause\par\padding
$d$ is called \b{reversal distance} between $\sigma$ and $\tau$.
\end{definition}

\end{frame}

\section{MIN-SBR}

\begin{frame}

Observation: The reversal distance between $\sigma$ and $\tau$ is the same as the reversal distance between $\tau^{-1} \circ \sigma$ and $id$. \pause\padding

\begin{definition}[MIN-SBR]
Let $\pi = \tau^{-1} \circ \sigma \in S_n$. \par\pause
\textit{Sorting by Reversals} is the problem of finding a sequence of reversals $\rho_1, \dots, \rho_d \in S_n$ such that
\begin{align*}
    \pi \circ \rho_1 \circ \cdots \circ \rho_d = id
\end{align*}
and $d$ is minimal.
\end{definition}

\end{frame}

\begin{frame}

\begin{exampleblock}{Example}
Let $\pi = (0\ 1\ 3\ 4\ 2\ 5) \in S_4$.\pause
\begin{align*}
    \pi \circ \rho(2,4) &= (0\ 1\ \r{2\ 4\ 3}\ 5) \\ \pause
    \pi \circ \rho(2,4) \circ \rho(3,4) &= (0\ 1\ 2\ \r{3\ 4}\ 5) = id
\end{align*}\pause
$\implies d(\pi) \leq 2$.
\end{exampleblock}

% \begin{algorithm}[ratchet algorithm, \citeauthor*{Watterson19821}]
% For each element at position $i$, perform $\rho(i, \pi_i^{-1})$ if $i \neq \pi_i$.
% \end{algorithm}\pause

% \begin{corollary}[tight upper bound, \citeauthor*{Kececioglu1995}]
% $d(\pi) \leq n - 1$ for all $\pi \in S_n$.
% \end{corollary}

\end{frame}

\section{Breakpoint graph}

\begin{frame}

\begin{definition}
Let $i \sim j$ if $|i - j| = 1$. \par\pause
A pair of consecutive elements $\pi_i$ and $\pi_j$ is
\begin{itemize}
    \item an \b{adjacency} if $\pi_i \sim \pi_j$; and
    \item a \b{breakpoint} if $\pi_i \not\sim \pi_j$.
\end{itemize} \par\pause\spadding
$b(\pi)$ denotes the number of breakpoints in $\pi \in S_n$.
\end{definition}\pause\padding

Observation: $b(\pi) = 0$ iff $\pi = id$ \pause and any reversal can at most eliminate two breakpoints. \pause\spadding

\begin{corollary}[lower bound, \citeauthor*{Kececioglu1995}]
$d(\pi) \geq \left\lceil \frac{b(\pi)}{2} \right\rceil$ for all $\pi \in S_n$.
\end{corollary}

\end{frame}

\begin{frame}

\begin{definition}[breakpoint graph, \citeauthor*{Bafna1996}]
Let $G(\pi) = (V, E)$ with
\begin{itemize}
    \item vertices $V = [0, n+1]$ representing the elements of $\pi$\pause; and
    \item edges $E = \r R \cup \b B$ with
    \begin{itemize}
        \item a \r{red} edge for every breakpoint in $\pi$; and
        \item a \b{blue} edge for every missing adjacency in $\pi$.
    \end{itemize}
\end{itemize}
\end{definition}

\end{frame}

\begin{frame}

\begin{exampleblock}{Example}
Let $\pi = (0\ 1\ \r\mid\ 3\ 4\ \r\mid\ 2\ \r\mid\ 5) \in S_4$. \pause Then $G(\pi)$ is

\spadding\begin{center}
\begin{tikzpicture}
	\begin{pgfonlayer}{nodelayer}
		\node [style=node] (0) at (-3, 0) {0};
		\node [style=node] (1) at (-1.5, 0) {1};
		\node [style=node] (2) at (0, 0) {3};
		\node [style=node] (3) at (1.5, 0) {4};
		\node [style=node] (4) at (3, 0) {2};
		\node [style=node] (5) at (4.5, 0) {5};
	\end{pgfonlayer}
	\begin{pgfonlayer}{edgelayer}
		\draw [style=red edge] (1) to (2);
		\draw [style=red edge] (3) to (4);
		\draw [style=red edge] (4) to (5);
		\draw [style=blue edge, bend right] (1) to (4);
		\draw [style=blue edge, bend left] (2) to (4);
		\draw [style=blue edge, bend left] (3) to (5);
	\end{pgfonlayer}
\end{tikzpicture}
\end{center}
\end{exampleblock}\pause

Observation: Each vertex has an equal number of incident red and blue edges. \pause\spadding

\begin{corollary}[\citeauthor*{Bafna1996}]
$G(\pi)$ can be decomposed into edge-disjoint alternating cycles.
\end{corollary}

\end{frame}

\begin{frame}

\begin{center}
\begin{tikzpicture}[scale=0.6, every node/.style={scale=0.6}]
	\begin{pgfonlayer}{nodelayer}
		\node [style=node] (0) at (-3, 0) {0};
		\node [style=node] (1) at (-1.5, 0) {1};
		\node [style=node] (2) at (0, 0) {3};
		\node [style=node] (3) at (1.5, 0) {4};
		\node [style=node] (4) at (3, 0) {2};
		\node [style=node] (5) at (4.5, 0) {5};
	\end{pgfonlayer}
	\begin{pgfonlayer}{edgelayer}
		\draw [style=red edge] (1) to (2);
		\draw [style=red edge] (3) to (4);
		\draw [style=red edge] (4) to (5);
		\draw [style=blue edge, bend right] (1) to (4);
		\draw [style=blue edge, bend left] (2) to (4);
		\draw [style=blue edge, bend left] (3) to (5);
	\end{pgfonlayer}
\end{tikzpicture}
\end{center}

\begin{definition}
A reversal is called \b{$k$-reversal} if it removes $k$ breakpoints. \pause\spadding

A reversal \b{acts on} two red edges of $G(\pi)$ if those two edges represent the breakpoints that are split apart by the reversal. \pause\spadding

An alternating cycle in $G(\pi)$ is a \b{$k$-cycle} if it has $k$ constituting red edges. \pause\spadding

We call an alternating cycle $C$ in $G(\pi)$ \b{oriented} if there is a $1$- or $2$-reversal acting on two constituting red edges of $C$.
\end{definition}

\end{frame}

\begin{frame}

\begin{lemma}[\citeauthor*{Christie1998}]
\label{lem:2}
An oriented alternating cycle is a $2$-cycle iff the corresponding reversal acting on both red edges is a $2$-reversal.
\end{lemma}\pause

\begin{proof}
Let $i \sim i'$ and $j \sim j'$.

\begin{center}
\begin{tikzpicture}
	\begin{pgfonlayer}{nodelayer}
		\node [style=node] (0) at (-1.25, 0) {i};
		\node [style=node] (1) at (1.25, 0) {j'};
		\node [style=node] (2) at (3, 0) {i'};
		\node [style=node] (3) at (-3, 0) {j};
		\node (4) at (0, 0) {$\dots$};
		\node (5) at (-4.25, 0) {$\dots$};
		\node (6) at (4.25, 0) {$\dots$};
	\end{pgfonlayer}
	\begin{pgfonlayer}{edgelayer}
		\draw [style=blue edge, bend right] (3) to (1);
		\draw [style=blue edge, bend left] (0) to (2);
		\draw [style=red edge] (3) to (0);
		\draw [style=red edge] (1) to (2);
	\end{pgfonlayer}
\end{tikzpicture}
\end{center}
\end{proof}

\end{frame}

\begin{frame}

Let $c(\pi)$ denote the maximum number of alternating cycles in any alternating cycle decomposition of $G(\pi)$. \pause\padding

\begin{theorem}[\citeauthor*{Bafna1996}]
\label{thm:1}
Let $\pi, \rho \in S_n$ and $\rho$ be a reversal. \pause Then
\begin{align*}
    b(\pi) - b(\pi \circ \rho) + c(\pi \circ \rho) - c(\pi) \leq 1.
\end{align*}
\end{theorem}

\end{frame}

\begin{frame}

\begin{proof}
To show: $b(\pi) - b(\pi \circ \rho) + c(\pi \circ \rho) - c(\pi) \leq 1$. \par
We consider each case $b(\pi) - b(\pi \circ \rho) \in [-2, 2]$ separately.\pause
\begin{enumerate}
    \item A $2$-reversal removes at least one alternating cycle from the maximum alternating cycle decomposition.
        \begin{tabularx}{\textwidth}{XcX}
            \begin{tikzpicture}[scale=0.6, every node/.style={scale=0.6}]
            	\begin{pgfonlayer}{nodelayer}
            		\node [style=node] (0) at (-1.25, 0) {i};
            		\node [style=node] (1) at (1.25, 0) {j'};
            		\node [style=node] (2) at (3, 0) {i'};
            		\node [style=node] (3) at (-3, 0) {j};
            		\node (4) at (0, 0) {$\dots$};
            	\end{pgfonlayer}
            	\begin{pgfonlayer}{edgelayer}
            		\draw [style=blue edge, bend right] (3) to (1);
            		\draw [style=blue edge, bend left] (0) to (2);
            		\draw [style=red edge] (3) to (0);
            		\draw [style=red edge] (1) to (2);
            	\end{pgfonlayer}
            \end{tikzpicture} & $\rightarrow$ &
            \begin{tikzpicture}[scale=0.6, every node/.style={scale=0.6}]
            	\begin{pgfonlayer}{nodelayer}
            		\node [style=node] (0) at (-1.25, 0) {j'};
            		\node [style=node] (1) at (1.25, 0) {i};
            		\node [style=node] (2) at (3, 0) {i'};
            		\node [style=node] (3) at (-3, 0) {j};
            		\node (4) at (0, 0) {$\dots$};
            	\end{pgfonlayer}
            \end{tikzpicture}
        \end{tabularx}\pause

    \item A $1$-reversal does not add an alternating cycle to the maximum alternating cycle decomposition.
        \begin{tabularx}{\textwidth}{XcX}
            \begin{tikzpicture}[scale=0.6, every node/.style={scale=0.6}]
            	\begin{pgfonlayer}{nodelayer}
            		\node [style=node] (0) at (-1.25, 0) {i};
            		\node [style=node] (1) at (1.25, 0) {j'};
            		\node [style=node] (2) at (3, 0) {k};
            		\node [style=node] (3) at (-3, 0) {j};
            		\node (4) at (0, 0) {$\dots$};
            	\end{pgfonlayer}
            	\begin{pgfonlayer}{edgelayer}
            		\draw [style=blue edge, bend right] (3) to (1);
            		\draw [style=red edge] (3) to (0);
            		\draw [style=red edge] (1) to (2);
            		\draw [dashed, bend left] (0) to (2);
            	\end{pgfonlayer}
            \end{tikzpicture} & $\rightarrow$ &
            \begin{tikzpicture}[scale=0.6, every node/.style={scale=0.6}]
            	\begin{pgfonlayer}{nodelayer}
            		\node [style=node] (0) at (-1.25, 0) {j'};
            		\node [style=node] (1) at (1.25, 0) {i};
            		\node [style=node] (2) at (3, 0) {k};
            		\node [style=node] (3) at (-3, 0) {j};
            		\node (4) at (0, 0) {$\dots$};
            	\end{pgfonlayer}
            	\begin{pgfonlayer}{edgelayer}
            		\draw [style=red edge] (1) to (2);
            		\draw [dashed, bend left] (1) to (2);
            	\end{pgfonlayer}
            \end{tikzpicture}
        \end{tabularx}
\end{enumerate}\pause
Proof for other cases similar.
\end{proof}

\end{frame}

\begin{frame}

\begin{theorem}[lower bound, \citeauthor*{Bafna1996}]
\label{thm:2}
Let $\pi \in S_n$. Then
\begin{align*}
    d(\pi) \geq b(\pi) - c(\pi).
\end{align*}
\end{theorem}\pause

\begin{proof}\renewcommand{\qedsymbol}{}
Let $\pi_t = \pi, \pi_0 = id$ and $\rho_1, \dots, \rho_t$ a shortest sequence of reversals from $\pi_t$ to $\pi_0$. \par\pause
Then
\begin{align*}
    d(\pi_i) &= d(\pi_{i-1}) + 1 \\ \pause
             \overset{(\ref{thm:1})}&{\geq} d(\pi_{i-1}) + b(\pi_i) - b(\pi_{i-1}) + c(\pi_{i-1}) - c(\pi_i).
\end{align*}
\end{proof}

\end{frame}

\begin{frame}

\begin{proof}[Proof (cont.)]
Therefore
\begin{align*}
    d(\pi_i) - (b(\pi_i) - c(\pi_i)) &\geq d(\pi_{i-1}) - (b(\pi_{i-1}) - c(\pi_{i-1})) \\ \pause
                                     &\geq d(\pi_0) - (b(\pi_0) - c(\pi_0))\pause = 0
\end{align*}\pause
Let $i = t$. Then
\begin{align*}
    && d(\pi_t) - (b(\pi_t) - c(\pi_t)) &\geq 0 \\ \pause
    \iff && d(\pi) - (b(\pi) - c(\pi))  &\geq 0 \\ \pause
    \iff && d(\pi)                      &\geq b(\pi) - c(\pi).
\end{align*}
\end{proof}

\end{frame}

\begin{frame}

\begin{lemma}[\citeauthor*{Christie1998}]
\label{lem:1}
Let $C$ be the maximum alternating cycle decomposition of $G(\pi)$. \par\pause
Let $c_2(\pi)$ be the minimum number of alternating $2$-cycles in $C$ \pause and let $c_{3*}(\pi) = c(\pi) - c_2(\pi)$ be the remaining alternating cycles in $C$. \pause
Then
\begin{align*}
    c_{3*}(\pi) \leq \frac{1}{3}(b(\pi) - 2 c_2(\pi)).
\end{align*}
\end{lemma}

\begin{proof}
\begin{itemize}
    \item $b(\pi)$ is the number of red edges in $G(\pi)$ and $c_2(\pi)$ is the number of red edges in $2$-cycles of $G(\pi)$. \par\pause
    \item $b(\pi) - 2 c_2(\pi)$ is the number of red edges in $G(\pi)$ not occurring in $2$-cycles. \par\pause
    \item $3 c_{3*}(\pi)$ is a lower bound to the number of red edges in $G(\pi)$ not occurring in $2$-cycles.\qedhere
\end{itemize}
\end{proof}

\end{frame}

\begin{frame}

\begin{theorem}[lower bound with $2$-cycles, \citeauthor*{Christie1998}]
\label{thm:3}
Let $\pi \in S_n$. Then
\begin{align*}
    d(\pi) \geq \frac{2}{3} b(\pi) - \frac{1}{3} c_2(\pi).
\end{align*}
\end{theorem}\pause

\begin{proof}
\begin{align*}
    d(\pi) \overset{(\ref{thm:2})}&{\geq} b(\pi) - c(\pi) \\ \pause
           &\geq b(\pi) - c_2(\pi) - c_{3*}(\pi) \\ \pause
           \overset{(\ref{lem:1})}&{\geq} b(\pi) - c_2(\pi) - \frac{1}{3}(b(\pi) - 2 c_2(\pi)) \\ \pause
           &= \frac{2}{3} b(\pi) - \frac{1}{3} c_2(\pi).
\end{align*}
\end{proof}

\end{frame}

\section{3/2-approximation}

\begin{frame}

\begin{block}{$\frac{3}{2}$-approximation}
By theorem \ref{thm:3}, an algorithm that finds a sorting sequence of reversals of at most length $b(\pi) - \frac{1}{2} c_2(\pi)$ achieves an approximation bound of $\frac{3}{2}$. \pause\spadding

We find such an algorithm in two steps:
\begin{enumerate}
    \item given an alternating cycle decomposition $C$ of $G(\pi)$ we find a sorting sequence of reversals for $\pi$\pause; and
    \item we find an alternating cycle decomposition of $G(\pi)$ maximizing the number of $2$-cycles.
\end{enumerate}
\end{block}\pause

Lastly, we prove the approximation bound.

\end{frame}

\subsection{Reversal graph}

\begin{frame}

\begin{definition}[reversal graph, \citeauthor*{Christie1998}]
Given an alternating cycle decomposition $C$ of $G(\pi)$, \par\pause
define $R(C)$ with
\begin{itemize}
    \item an isolated \b{blue} vertex for each adjacency in $\pi$\pause;
    \item $m$ vertices for each $m$-cycle in $C$ each representing the reversal $\rho(u)$ acting on the two red edges connected by a blue edge\pause;
    \begin{itemize}
        \item a vertex is \r{red} if the represented reversal is a $1$- or $2$-reversal
        \item a vertex is \b{blue} otherwise
    \end{itemize}\pause
    \item connect two vertices with an edge if their corresponding blue edges \textit{interleave}.
\end{itemize}
\end{definition}\pause

Observation: The only alternating cycle decomposition of $G(id)$ is $C = \emptyset$.\pause

\begin{corollary}[\citeauthor*{Christie1998}]
$R(C)$ consists of $n+1$ isolated blue vertices.
\end{corollary}

\end{frame}

\begin{frame}

\begin{exampleblock}{Example}
Let $\pi = (0\ 4\ 2\ 1\ 3\ 5) \in S_4$. \par
Given the alternating cycle decomposition $C$ of $G(\pi)$

\begin{tabularx}{\textwidth}{XR}
{\footnotesize\begin{align*}
    C = \{(&\{1,3\},\{2,3\},\{2,4\}, \\
           &\{4,5\},\{2,5\},\{1,2\})\}
\end{align*}} &
\begin{tikzpicture}[scale=0.6, every node/.style={scale=0.6}]
	\begin{pgfonlayer}{nodelayer}
		\node [style=node] (0) at (-3, 0) {0};
		\node [style=node] (1) at (-1.5, 0) {1};
		\node [style=node] (2) at (0, 0) {3};
		\node [style=node] (3) at (1.5, 0) {4};
		\node [style=node] (4) at (3, 0) {2};
		\node [style=node] (5) at (4.5, 0) {5};
	\end{pgfonlayer}
	\begin{pgfonlayer}{edgelayer}
		\draw [style=red edge] (1) to (2);
		\draw [style=red edge] (3) to (4);
		\draw [style=red edge] (4) to (5);
		\draw [style=blue edge, bend right] (1) to (4);
		\draw [style=blue edge, bend left] (2) to (4);
		\draw [style=blue edge, bend left] (3) to (5);
	\end{pgfonlayer}
\end{tikzpicture}
\end{tabularx}\pause

construct $R(C)$.

\begin{center}
\begin{tikzpicture}[scale=0.8, every node/.style={scale=0.8}]
	\begin{pgfonlayer}{nodelayer}
		\node [style=red node] (0) at (0, 1) {$\{2,3\}$};
		\node [style=red node] (1) at (-2, 1) {$\{1,2\}$};
		\node [style=blue node] (2) at (-1, -1) {$\{4,5\}$};
		\node [style=blue node] (3) at (2, -1) {$\{3,4\}$};
		\node [style=blue node] (4) at (2, 1) {$\{0,1\}$};
	\end{pgfonlayer}
	\begin{pgfonlayer}{edgelayer}
		\draw (0) to (2);
		\draw (1) to (2);
	\end{pgfonlayer}
\end{tikzpicture}
\end{center}
\end{exampleblock}\pause

Idea: Each connected component of $R(C)$ can be sorted separately.

\end{frame}

\begin{frame}

Let $u$ be a vertex of $R(C)$.\pause

\begin{definition}
Denote by $C_u$ the alternating cycle decomposition of $G(\pi \circ \rho(u))$ that is obtained from $C$ by removing one 2-cycle or shortening one of its cycles by one red edge.
\end{definition}\pause

\begin{lemma}[\citeauthor*{Christie1998}]
\label{lem:5}
$R(C_u)$ can be derived from $R(C)$ by making the following changes to $R(C)$:\pause
\begin{enumerate}
    \item flip the color of every vertex adjacent to $u$\pause;
    \item flip the adjacency of every pair of vertices adjacent to $u$\pause; and
    \item if $u$ is a red vertex, turn it into an isolated blue vertex.
\end{enumerate}
\end{lemma}

\end{frame}

\begin{frame}

\begin{exampleblock}{Example}
Let $\pi = (0\ 1\ 3\ 4\ 2\ 5) \in S_4$ and $u = \{1,2\}$.\padding

\begin{tabularx}{\textwidth}{X|R}
$G(\pi)$ and $R(C)$ & $G(\pi \circ \rho(u))$ and $R(C_u)$ \\ \hline
\spadding\begin{tikzpicture}[scale=0.6, every node/.style={scale=0.6}]
	\begin{pgfonlayer}{nodelayer}
		\node [style=node] (0) at (-3, 0) {0};
		\node [style=node] (1) at (-1.5, 0) {1};
		\node [style=node] (2) at (0, 0) {3};
		\node [style=node] (3) at (1.5, 0) {4};
		\node [style=node] (4) at (3, 0) {2};
		\node [style=node] (5) at (4.5, 0) {5};
	\end{pgfonlayer}
	\begin{pgfonlayer}{edgelayer}
		\draw [style=red edge] (1) to (2);
		\draw [style=red edge] (3) to (4);
		\draw [style=red edge] (4) to (5);
		\draw [style=blue edge, bend right] (1) to (4);
		\draw [style=blue edge, bend left] (2) to (4);
		\draw [style=blue edge, bend left] (3) to (5);
	\end{pgfonlayer}
\end{tikzpicture} &
\onslide<2->{\begin{tikzpicture}[scale=0.6, every node/.style={scale=0.6}]
	\begin{pgfonlayer}{nodelayer}
		\node [style=node] (0) at (-3, 0) {0};
		\node [style=node] (1) at (-1.5, 0) {1};
		\node [style=node] (2) at (0, 0) {2};
		\node [style=node] (3) at (1.5, 0) {4};
		\node [style=node] (4) at (3, 0) {3};
		\node [style=node] (5) at (4.5, 0) {5};
	\end{pgfonlayer}
	\begin{pgfonlayer}{edgelayer}
		\draw [style=red edge] (2) to (3);
		\draw [style=red edge] (4) to (5);
		\draw [style=blue edge, bend left] (2) to (4);
		\draw [style=blue edge, bend left] (3) to (5);
	\end{pgfonlayer}
\end{tikzpicture}}\\
\spadding\begin{tikzpicture}[scale=0.8, every node/.style={scale=0.8}]
	\begin{pgfonlayer}{nodelayer}
		\node [style=red node] (0) at (0, 1) {$\{2,3\}$};
		\node [style=red node] (1) at (-2, 1) {$\{1,2\}$};
		\node [style=blue node] (2) at (-1, -1) {$\{4,5\}$};
		\node [style=blue node] (3) at (2, -1) {$\{3,4\}$};
		\node [style=blue node] (4) at (2, 1) {$\{0,1\}$};
	\end{pgfonlayer}
	\begin{pgfonlayer}{edgelayer}
		\draw (0) to (2);
		\draw (1) to (2);
	\end{pgfonlayer}
\end{tikzpicture} &
\onslide<3->{\spadding\begin{tikzpicture}[scale=0.8, every node/.style={scale=0.8}]
	\begin{pgfonlayer}{nodelayer}
		\node [style=red node] (0) at (0, 1) {$\{2,3\}$};
		\node [style=blue node] (1) at (-2, 1) {$\{1,2\}$};
		\node [style=red node] (2) at (-1, -1) {$\{4,5\}$};
		\node [style=blue node] (3) at (2, -1) {$\{3,4\}$};
		\node [style=blue node] (4) at (2, 1) {$\{0,1\}$};
	\end{pgfonlayer}
	\begin{pgfonlayer}{edgelayer}
		\draw (0) to (2);
	\end{pgfonlayer}
\end{tikzpicture}}
\end{tabularx}
\end{exampleblock}

\end{frame}

\begin{frame}

\begin{corollary}[\citeauthor*{Christie1998}]
A reversal $\rho(u)$ affects only vertices that are in the same connected component as $u$.
\end{corollary}\pause

\begin{lemma}[\citeauthor*{Christie1998}]
\label{lem:3}
All vertices arising from the same alternating cycle in $C$ are in the same connected component of $R(C)$.
\end{lemma}

\begin{corollary}[\citeauthor*{Christie1998}]
$R(C)$ contains no isolated blue vertices arising from cycles in $C$.
\end{corollary}

\end{frame}

\begin{frame}

\begin{lemma}[\citeauthor*{Christie1998}]
\label{lem:7}
Vertices arising from an unoriented $2$-cycle of $C$ must be in a connected component of $R(C)$ with vertices arising from at least one more alternating cycle of $C$.
\end{lemma}

\begin{center}
\begin{tikzpicture}[scale=0.8, every node/.style={scale=0.8}]
	\begin{pgfonlayer}{nodelayer}
		\node [style=node] (0) at (-3, 0) {};
		\node [style=node] (1) at (-1.25, 0) {};
		\node [style=node] (2) at (1.25, 0) {};
		\node [style=node] (3) at (3, 0) {};
		\node (4) at (0, 0) {$\dots$};
		\node (5) at (-4.25, 0) {$\dots$};
		\node (6) at (4.25, 0) {$\dots$};
	\end{pgfonlayer}
	\begin{pgfonlayer}{edgelayer}
		\draw [style=blue edge, bend left] (0) to node[sloped,midway,above] {u} (3);
		\draw  [style=blue edge, bend right](1) to node[sloped,midway,below] {v} (2);
		\draw [style=red edge] (0) to (1);
		\draw [style=red edge] (2) to (3);
	\end{pgfonlayer}
\end{tikzpicture}
\end{center}\pause

\begin{proof}
Let $u$ and $v$ represent the blue edges of an unoriented $2$-cycle of $C$. \par\pause
Then its two constituting blue edges do not interleave. Hence, $R(C)$ does not connect $u$ and $v$ with an edge. \par\pause
If its constituting blue edges do not interleave with blue edges of any other alternating cycle of $C$, then $u$ and $v$ would be isolated blue vertices. \b{$\lightning$}
\end{proof}

\end{frame}

\begin{frame}

\begin{definition}
We call a connected component of $R(C)$ \b{oriented} if it contains a red vertex or if it consists solely of an isolated blue vertex. \pause\spadding

Let $A$ be a connected component of $R(C)$. We denote by $A_u$ the subgraph of $R(C_u)$ that contains all the vertices of $A$.
\end{definition}\pause

\begin{lemma}[\citeauthor*{Christie1998}]
\label{lem:4}
If a connected component $A$ of $R(C)$ is oriented and not an isolated blue vertex, it contains a red vertex $u$ such that $A_u$ is still oriented.
\end{lemma}

\end{frame}

\begin{frame}

% \begin{definition}
% A sequence of reversals that turns the connected component $A$ of $R(C)$ into an isolated blue vertex, is called an \b{elemination sequence} for $A$.
% \end{definition}\pause

\begin{lemma}[\citeauthor*{Christie1998}]
\label{lem:6}
Let $A$ be an oriented connected component of $R(C)$ arising from $k$ different cycles of $G(\pi)$. Then there is an \textit{elimination sequence} for $A$ that contains $k$ $2$-reversals with all the other reversals being $1$-reversals.
\end{lemma}\pause

\begin{proof}
We know that
\begin{itemize}
    \item every vertex in $R(C)$ arising from an alternating cycle of $G(\pi)$ is contained within the same connected component (lemma \ref{lem:3})\pause;
    \item  as $A$ is oriented, we can use lemma \ref{lem:4} to repeatedly find a red vertex $u$ in $A$ and apply $\rho(u)$\pause; and
    \item $\rho(u)$ is a $2$-reversal iff $u$ arises from a $2$-cycle (lemma \ref{lem:2}).
\end{itemize}\pause
Every alternating cycle represented by vertices constituting $A$ eventually reduces to a $2$-cycle.
\end{proof}

\end{frame}

\begin{frame}

\begin{lemma}[\citeauthor*{Christie1998}]
Let $A$ be an unoriented connected component of $R(C)$ arising from $k$ different cycles of $G(\pi)$. Then there is an \textit{elimination sequence} for $A$ that contains one $0$-reversal, $k$ $2$-reversals, with all the other reversals being $1$-reversals.
\end{lemma}\pause

\begin{proof}
Let $\rho(u)$ be a $0$-reversal. Then $A_u$ is oriented (lemma \ref{lem:5}). \par\pause
Now apply lemma \ref{lem:6} to obtain an elimination sequence for $A_u$.
\end{proof}\pause

\begin{theorem}[\citeauthor*{Christie1998}]
\label{thm:4}
Let $\pi \in S_n$ and $C$ be a cycle decomposition of $G(\pi)$. Then
\begin{align*}
    d(\pi) \leq b(\pi) - |C| + u(C)
\end{align*}
where $u(C)$ is the number of unoriented components in $R(C)$.
\end{theorem}

\end{frame}

\subsection{Matching graph}

\begin{frame}

Goal: Find a cycle decomposition of $G(\pi)$ that has a large number of $2$-cycles.\pause

\begin{idea}
\begin{enumerate}
    \item Construct a \b{matching graph} $F(\pi)$ where vertices represent red edges in $G(\pi)$ and vertices $u, v$ are adjacent if they form a $2$-cycle in $G(\pi)$. \par\pause
    \item Find maximum cardinality matching $M$ of $F(\pi)$.\pause
    \item Use a \b{ladder graph} $L(M)$ with vertices representing $2$-cycles in $M$ and form connected components (\textit{ladders}) with $2$-cycles sharing a blue edge in $G(\pi)$.
\end{enumerate}
\end{idea}

\end{frame}

\begin{frame}

\begin{definition}
We call a $2$-cycle \b{selected} if its corresponding edge of $F(\pi)$ is in $M$.\pause\spadding

A selected $2$-cycle is called \b{independent} if it is not part of a ladder. Otherwise it is called a \b{ladder $2$-cycle}.
\end{definition}

\end{frame}

\begin{frame}

Let $z$ be the number of independent $2$-cycles, and $y$ the number of ladder $2$-cycles.\pause

\begin{theorem}[\citeauthor*{Christie1998}]
\label{thm:5}
Given a maximum cardinality matching $M$ of $F(\pi)$ it is possible to find a cycle decomposition $C$ of $G(\pi)$ that contains at least $\left\lceil \frac{y}{2} \right\rceil$ ladder $2$-cycles and $z$ independent $2$-cycles.
\end{theorem}

\begin{proof}
Let $C$ contain all independent $2$-cycles from $L(M)$ and every alternate cycle of each ladder. \par\pause
Obtain the rest of $C$ by adding any alternating cycle decomposition of the remaining unused edges of $G(\pi)$.
\end{proof}

\end{frame}

\subsection{Approximation bound}

\begin{frame}

\begin{theorem}[\citeauthor*{Christie1998}]
\label{thm:6}
Let $\pi \in S_n$. Then
\begin{align*}
    d(\pi) \leq b(\pi) - \frac{1}{2} c_2(\pi).
\end{align*}
\end{theorem}\pause

\begin{proof}\renewcommand{\qedsymbol}{}
Using theorem \ref{thm:5}, first find an alternating cycle decomposition $C$ of $G(\pi)$ with at least $\left\lceil \frac{y}{2} \right\rceil$ $2$-cycles as part of ladders and $z$ independent $2$-cycles.
\end{proof}

\end{frame}

\begin{frame}

\begin{proof}[Proof (cont.)]\renewcommand{\qedsymbol}{}
\begin{itemize}
    \item Let $k$ be the number of $2$-cycles in oriented connected components of $R(C)$.\pause
    \item Let $u$ be the number of unoriented connected components in $R(C)$ that include $l$ selected $2$-cycles and that contain vertices representing remaining unselected $2$-cycles.\pause
    \item Let $v$ be the number of remaining unoriented connected components consisting only of vertices representing $m$ independent selected $2$-cycles.
\end{itemize}\pause
By theorem \ref{thm:4}, we can sort $\pi$ using at least $k+l+u+m$ $2$-reversals and only $u+v$ $0$-reversals. \pause Therefore
\begin{align*}
    d(\pi) &\leq b(\pi) - k - l - u - m + u + v \\
           &= b(\pi) - k - l - m + v
\end{align*}
\end{proof}

\end{frame}

\begin{frame}

\begin{proof}[Proof (cont.)]
We know that
\begin{enumerate}
    \item $k+l+m \geq \left\lceil \frac{y}{2} \right\rceil + z$ as $\left\lceil \frac{y}{2} \right\rceil + z$ is the number of selected $2$-cycles by $M$\pause;
    \item $v \leq \left\lceil \frac{z}{2} \right\rceil$ as every unoriented component representing a $2$-cycle represents at least one more alternating cycle (lemma \ref{lem:7})\pause; and
    \item $|M| = y+z \geq c_2(\pi)$.
\end{enumerate}\pause
\begin{align*}
    k+l+m-v &\geq \left\lceil \frac{y}{2} \right\rceil + z - v                                          & (1) \\ \pause
            &\geq \left\lceil \frac{y}{2} \right\rceil + z - \left\lceil \frac{z}{2} \right\rceil       & (2) \\ \pause
            &= \left\lceil \frac{y}{2} \right\rceil + \left\lceil \frac{z}{2} \right\rceil \\ \pause
            &\geq \frac{c_2(\pi)}{2}                                                                    & (3)
\end{align*}
\end{proof}

\end{frame}

\begin{frame}[allowframebreaks]{References}

\printbibliography

\end{frame}

\end{document}
